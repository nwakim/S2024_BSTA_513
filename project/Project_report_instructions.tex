% Options for packages loaded elsewhere
\PassOptionsToPackage{unicode}{hyperref}
\PassOptionsToPackage{hyphens}{url}
\PassOptionsToPackage{dvipsnames,svgnames,x11names}{xcolor}
%
\documentclass[
  letterpaper,
  DIV=11,
  numbers=noendperiod]{scrartcl}

\usepackage{amsmath,amssymb}
\usepackage{iftex}
\ifPDFTeX
  \usepackage[T1]{fontenc}
  \usepackage[utf8]{inputenc}
  \usepackage{textcomp} % provide euro and other symbols
\else % if luatex or xetex
  \usepackage{unicode-math}
  \defaultfontfeatures{Scale=MatchLowercase}
  \defaultfontfeatures[\rmfamily]{Ligatures=TeX,Scale=1}
\fi
\usepackage{lmodern}
\ifPDFTeX\else  
    % xetex/luatex font selection
\fi
% Use upquote if available, for straight quotes in verbatim environments
\IfFileExists{upquote.sty}{\usepackage{upquote}}{}
\IfFileExists{microtype.sty}{% use microtype if available
  \usepackage[]{microtype}
  \UseMicrotypeSet[protrusion]{basicmath} % disable protrusion for tt fonts
}{}
\makeatletter
\@ifundefined{KOMAClassName}{% if non-KOMA class
  \IfFileExists{parskip.sty}{%
    \usepackage{parskip}
  }{% else
    \setlength{\parindent}{0pt}
    \setlength{\parskip}{6pt plus 2pt minus 1pt}}
}{% if KOMA class
  \KOMAoptions{parskip=half}}
\makeatother
\usepackage{xcolor}
\setlength{\emergencystretch}{3em} % prevent overfull lines
\setcounter{secnumdepth}{-\maxdimen} % remove section numbering
% Make \paragraph and \subparagraph free-standing
\ifx\paragraph\undefined\else
  \let\oldparagraph\paragraph
  \renewcommand{\paragraph}[1]{\oldparagraph{#1}\mbox{}}
\fi
\ifx\subparagraph\undefined\else
  \let\oldsubparagraph\subparagraph
  \renewcommand{\subparagraph}[1]{\oldsubparagraph{#1}\mbox{}}
\fi


\providecommand{\tightlist}{%
  \setlength{\itemsep}{0pt}\setlength{\parskip}{0pt}}\usepackage{longtable,booktabs,array}
\usepackage{calc} % for calculating minipage widths
% Correct order of tables after \paragraph or \subparagraph
\usepackage{etoolbox}
\makeatletter
\patchcmd\longtable{\par}{\if@noskipsec\mbox{}\fi\par}{}{}
\makeatother
% Allow footnotes in longtable head/foot
\IfFileExists{footnotehyper.sty}{\usepackage{footnotehyper}}{\usepackage{footnote}}
\makesavenoteenv{longtable}
\usepackage{graphicx}
\makeatletter
\def\maxwidth{\ifdim\Gin@nat@width>\linewidth\linewidth\else\Gin@nat@width\fi}
\def\maxheight{\ifdim\Gin@nat@height>\textheight\textheight\else\Gin@nat@height\fi}
\makeatother
% Scale images if necessary, so that they will not overflow the page
% margins by default, and it is still possible to overwrite the defaults
% using explicit options in \includegraphics[width, height, ...]{}
\setkeys{Gin}{width=\maxwidth,height=\maxheight,keepaspectratio}
% Set default figure placement to htbp
\makeatletter
\def\fps@figure{htbp}
\makeatother

\KOMAoption{captions}{tableheading}
\makeatletter
\@ifpackageloaded{tcolorbox}{}{\usepackage[skins,breakable]{tcolorbox}}
\@ifpackageloaded{fontawesome5}{}{\usepackage{fontawesome5}}
\definecolor{quarto-callout-color}{HTML}{909090}
\definecolor{quarto-callout-note-color}{HTML}{0758E5}
\definecolor{quarto-callout-important-color}{HTML}{CC1914}
\definecolor{quarto-callout-warning-color}{HTML}{EB9113}
\definecolor{quarto-callout-tip-color}{HTML}{00A047}
\definecolor{quarto-callout-caution-color}{HTML}{FC5300}
\definecolor{quarto-callout-color-frame}{HTML}{acacac}
\definecolor{quarto-callout-note-color-frame}{HTML}{4582ec}
\definecolor{quarto-callout-important-color-frame}{HTML}{d9534f}
\definecolor{quarto-callout-warning-color-frame}{HTML}{f0ad4e}
\definecolor{quarto-callout-tip-color-frame}{HTML}{02b875}
\definecolor{quarto-callout-caution-color-frame}{HTML}{fd7e14}
\makeatother
\makeatletter
\makeatother
\makeatletter
\makeatother
\makeatletter
\@ifpackageloaded{caption}{}{\usepackage{caption}}
\AtBeginDocument{%
\ifdefined\contentsname
  \renewcommand*\contentsname{Table of contents}
\else
  \newcommand\contentsname{Table of contents}
\fi
\ifdefined\listfigurename
  \renewcommand*\listfigurename{List of Figures}
\else
  \newcommand\listfigurename{List of Figures}
\fi
\ifdefined\listtablename
  \renewcommand*\listtablename{List of Tables}
\else
  \newcommand\listtablename{List of Tables}
\fi
\ifdefined\figurename
  \renewcommand*\figurename{Figure}
\else
  \newcommand\figurename{Figure}
\fi
\ifdefined\tablename
  \renewcommand*\tablename{Table}
\else
  \newcommand\tablename{Table}
\fi
}
\@ifpackageloaded{float}{}{\usepackage{float}}
\floatstyle{ruled}
\@ifundefined{c@chapter}{\newfloat{codelisting}{h}{lop}}{\newfloat{codelisting}{h}{lop}[chapter]}
\floatname{codelisting}{Listing}
\newcommand*\listoflistings{\listof{codelisting}{List of Listings}}
\makeatother
\makeatletter
\@ifpackageloaded{caption}{}{\usepackage{caption}}
\@ifpackageloaded{subcaption}{}{\usepackage{subcaption}}
\makeatother
\makeatletter
\@ifpackageloaded{tcolorbox}{}{\usepackage[skins,breakable]{tcolorbox}}
\makeatother
\makeatletter
\@ifundefined{shadecolor}{\definecolor{shadecolor}{rgb}{.97, .97, .97}}
\makeatother
\makeatletter
\makeatother
\makeatletter
\makeatother
\ifLuaTeX
  \usepackage{selnolig}  % disable illegal ligatures
\fi
\IfFileExists{bookmark.sty}{\usepackage{bookmark}}{\usepackage{hyperref}}
\IfFileExists{xurl.sty}{\usepackage{xurl}}{} % add URL line breaks if available
\urlstyle{same} % disable monospaced font for URLs
\hypersetup{
  pdftitle={Project Report Instructions},
  pdfauthor={Nicky Wakim},
  colorlinks=true,
  linkcolor={blue},
  filecolor={Maroon},
  citecolor={Blue},
  urlcolor={Blue},
  pdfcreator={LaTeX via pandoc}}

\title{Project Report Instructions}
\usepackage{etoolbox}
\makeatletter
\providecommand{\subtitle}[1]{% add subtitle to \maketitle
  \apptocmd{\@title}{\par {\large #1 \par}}{}{}
}
\makeatother
\subtitle{BSTA 513/613}
\author{Nicky Wakim}
\date{}

\begin{document}
\maketitle
\ifdefined\Shaded\renewenvironment{Shaded}{\begin{tcolorbox}[interior hidden, enhanced, boxrule=0pt, borderline west={3pt}{0pt}{shadecolor}, frame hidden, sharp corners, breakable]}{\end{tcolorbox}}\fi

\hypertarget{directions}{%
\subsection{Directions}\label{directions}}

\begin{tcolorbox}[enhanced jigsaw, left=2mm, bottomrule=.15mm, coltitle=black, opacitybacktitle=0.6, title=\textcolor{quarto-callout-important-color}{\faExclamation}\hspace{0.5em}{Project template}, opacityback=0, arc=.35mm, colback=white, toprule=.15mm, toptitle=1mm, colbacktitle=quarto-callout-important-color!10!white, bottomtitle=1mm, rightrule=.15mm, titlerule=0mm, leftrule=.75mm, colframe=quarto-callout-important-color-frame, breakable]

You \emph{may} use
\href{https://github.com/nwakim/S2024_BSTA_513/blob/main/project/Project_template.qmd}{this
project template} to get started on the report. It is your
responsibility to meet the formatting guidelines below!!

DO NOT USE SITE PAGE (``Project Report Instructions'', current page) as
your template!!

\end{tcolorbox}

\hypertarget{purpose}{%
\subsubsection{Purpose}\label{purpose}}

Project reports serve as a great way to communicate the knowledge
learned in our class and connect it to context within research. It is
important that we can take a step back from the numbers and analysis to
see what questions linear regression can help us answer.

\textbf{It is really important for you to look back through your labs
and connect the work!!}

\hypertarget{formatting-guide}{%
\subsubsection{Formatting guide}\label{formatting-guide}}

\begin{itemize}
\tightlist
\item
  The report will be written in Quarto. Turn in both the \texttt{qmd}
  and \texttt{html} files

  \begin{itemize}
  \tightlist
  \item
    \textbf{No code} should appear in the \texttt{html} document

    \begin{itemize}
    \tightlist
    \item
      This means all R code chunks should have
      \texttt{\#\textbar{}\ echo:\ false}
    \item
      This also means warnings and messages should be turned off
    \end{itemize}
  \end{itemize}
\item
  The report should be 10 - 14 paragraphs long

  \begin{itemize}
  \tightlist
  \item
    \textbf{In 512, many lab reports were a little too long!}
  \item
    \textbf{Remember, I know mostly what you did in the labs! This
    report is meant to synthesize that work into a coherent
    message/story!}

    \begin{itemize}
    \tightlist
    \item
      \textbf{This often means details of our analysis are lost.}
    \end{itemize}
  \end{itemize}
\item
  Tables and figures should NOT have variable names as they appear in
  the data frame

  \begin{itemize}
  \tightlist
  \item
    Variable names should be understood by a reader
  \item
    Variable names should be written in full words
  \item
    Include a title or caption for all figures
  \item
    Figure and tables appear on same page or close to same page where
    they are first referenced
  \item
    Tables and figures are an appropriate size in the html - Nicky is
    able to read all words in figures and tables
  \end{itemize}
\item
  Writing, spelling, and grammar should be admissable

  \begin{itemize}
  \tightlist
  \item
    This means I can generally follow your thought/what you are trying
    to communicate
  \item
    Some spelling and grammar mistakes are allowed

    \begin{itemize}
    \tightlist
    \item
      I will not take off points if there are a few sprinkled in
    \item
      If \emph{every or close to every sentence} has mistakes, then I
      will take off
    \end{itemize}
  \end{itemize}
\item
  Sectioning of the report

  \begin{itemize}
  \tightlist
  \item
    Main sections that are required: Introduction, Statistical Methods,
    Results, Discussion, Conclusion, \textbf{Reflection}, and References

    \begin{itemize}
    \tightlist
    \item
      You may have an appendix to include additional figures!
    \end{itemize}
  \item
    Other sections that might help group specific methods or results
  \end{itemize}
\item
  Title information at the top of the \texttt{html}

  \begin{itemize}
  \tightlist
  \item
    This includes the title itself, your name, and the date
  \end{itemize}
\end{itemize}

\begin{tcolorbox}[enhanced jigsaw, left=2mm, bottomrule=.15mm, coltitle=black, opacitybacktitle=0.6, title=\textcolor{quarto-callout-note-color}{\faInfo}\hspace{0.5em}{The project report is a separate file from the labs}, opacityback=0, arc=.35mm, colback=white, toprule=.15mm, toptitle=1mm, colbacktitle=quarto-callout-note-color!10!white, bottomtitle=1mm, rightrule=.15mm, titlerule=0mm, leftrule=.75mm, colframe=quarto-callout-note-color-frame, breakable]

You can save tables and figures from labs or separate files, then load
them in the report

\begin{itemize}
\tightlist
\item
  Save R objects in analyses file:

  \begin{itemize}
  \tightlist
  \item
    Suppose you named the Table 1 as \texttt{table1}
  \item
    \texttt{save(table1,\ file\ =\ "table1.Rdata")}
  \end{itemize}
\item
  Load R objects in report file: \texttt{load(file\ =\ "table1.Rdata")}
\end{itemize}

\end{tcolorbox}

\hypertarget{examples-of-reports}{%
\subsubsection{Examples of reports}\label{examples-of-reports}}

The following are examples of reports from BSTA 513 with the feedback
that I gave them.

Please note that 513 uses a different type of outcome than our class.
These examples are meant to help guide you with the formatting and some
appropriate content.

Also note that these were converted to PDFs so I could write in
feedback. Some of the tables and figure sizes were distorted. They need
to be legible in the \texttt{html}.

\begin{itemize}
\item
  \href{https://ohsuitg-my.sharepoint.com/:b:/r/personal/wakim_ohsu_edu/Documents/Teaching/Classes/W2024_BSTA_512_612/Student_files/Project_examples/Group_09_Report.pdf?csf=1\&web=1\&e=54lAbD}{Report
  1 with my feedback}
\item
  \href{https://ohsuitg-my.sharepoint.com/:b:/r/personal/wakim_ohsu_edu/Documents/Teaching/Classes/W2024_BSTA_512_612/Student_files/Project_examples/Group_07_Report.pdf?csf=1\&web=1\&e=uWHgnB}{Report
  2 with my feedback}
\end{itemize}

The above reports have code showing in their \texttt{html}. Remember
that I am asking you to \textbf{hide} all code, warnings, and messages.

\subsubsection{Grading}

The project report is out of 36 points. Note that the Statistical
Methods and Results sections are graded on an 8-point scale, while all
other components are graded on a 4-point scale.

\subsubsection{Rubric}

\begin{longtable}[]{@{}
  >{\raggedright\arraybackslash}p{(\columnwidth - 10\tabcolsep) * \real{0.1667}}
  >{\raggedright\arraybackslash}p{(\columnwidth - 10\tabcolsep) * \real{0.1667}}
  >{\raggedright\arraybackslash}p{(\columnwidth - 10\tabcolsep) * \real{0.1667}}
  >{\raggedright\arraybackslash}p{(\columnwidth - 10\tabcolsep) * \real{0.1667}}
  >{\raggedright\arraybackslash}p{(\columnwidth - 10\tabcolsep) * \real{0.1667}}
  >{\raggedright\arraybackslash}p{(\columnwidth - 10\tabcolsep) * \real{0.1667}}@{}}
\toprule\noalign{}
\begin{minipage}[b]{\linewidth}\raggedright
\end{minipage} & \begin{minipage}[b]{\linewidth}\raggedright
4 points
\end{minipage} & \begin{minipage}[b]{\linewidth}\raggedright
3 points
\end{minipage} & \begin{minipage}[b]{\linewidth}\raggedright
2 points
\end{minipage} & \begin{minipage}[b]{\linewidth}\raggedright
1 point
\end{minipage} & \begin{minipage}[b]{\linewidth}\raggedright
0 points
\end{minipage} \\
\midrule\noalign{}
\endhead
\bottomrule\noalign{}
\endlastfoot
Formatting & Lab submitted on Sakai (or by email if late) with
\texttt{.html} file. Report is written in complete sentences with very
few grammatical or spelling errors. With little editing, the report can
be distributed. & Lab submitted on Sakai (or by email if late) with
\texttt{.html} file. Report is written in complete sentences with some
(around 2 per section) grammatical or spelling errors. With some
editing, the report can be distributed. & Lab submitted on Sakai (or by
email if late) with \texttt{.html} file. Report is written in complete
sentences, but have many grammatical or spelling errors. With major
editing, the report can be distributed. & Lab submitted on Sakai (or by
email if late) with \texttt{.html} file. Report is written in complete
sentences, but are very hard to follow due to grammar mistakes. & Lab
not submitted on Sakai (or by email if late) with \texttt{.html} file.
Report is not written with complete sentences. With major editing, the
report can be distributed. \\
Figures and work & All requested output is displayed, including 2
required figures and tables, and at least one additional figure. Figures
and tables look professional, are easily interpreted by the reader, and
easily convey the intended message. & All requested output is displayed,
including 2 required figures and tables, and at least one additional
figure. For the most part, figures and tables look professional, are
easily interpreted by the reader, and easily convey the intended
message. A few mistakes in the figures are made. & All requested output
is displayed, including 2 required figures and tables, and at least one
additional figure. Figures and tables look semi-professional, are not so
easily interpreted by the reader, and convey the intended message but
after some work by the reader. Some mistakes in the figures are made. &
All requested output is displayed, including 2 required figures and
tables, and at least one additional figure. Figures and tables do not
look professional, are not easily interpreted by the reader, and/or do
not convey the intended message. Many mistakes in the figures are made.
& Requested output is not displayed, Missing one or more figures. \\
Introduction & Provides a good background for the research question,
includes motivation for the question, and references previous research
that justifies this analysis. & Provides a decent background for the
research question and includes motivation for the question. Previous
research is mentioned, but feels disconnected to the current analysis. &
Provides a decent background for the research question and includes
motivation for the question. Previous research is mentioned, but feels
disconnected to the current analysis. & Does not provide a background
that connects to the research question. Motivation and previous research
are not mentioned. & No introduction included. \\
Methods (8 points) & Describes statistical methods concisely and
highlights pertinent information to the reader (listed Sections below).
Demonstrates proper analyses were performed. & Describes statistical
methods and highlights pertinent information to the reader (listed
Sections below). Details were omitted or added that were not needed to
explain the overarching methods. Demonstrates proper analyses were
performed. & Describes statistical methods and highlights pertinent
information to the reader (listed Sections below). Details were omitted
or added that were not needed to explain the overarching methods. Some
incorrect analyses included in the description. & Describes statistical
methods, but lacks clarity. Demonstrates a lack of understanding about
the overall process of regression analysis. Incorrect analyses included
in the description. & No methods included. \\
Results (8 points) & Correctly interprets coefficients for the
explanatory variable and identifies any other interesting trends.
Highlights pertinent results to the reader (listed Sections below). &
Correctly interprets coefficients, but does correctly incorporate the
interaction (if in the model). Highlights pertinent results to the
reader (listed Sections below). & Incorrectly interprets coefficients.
Highlights pertinent results to the reader (listed Sections below). &
Incorrectly interprets coefficients.Omits pertinent results to the
reader (listed Sections below). & No results included. \\
Discussion & Thoroughly and concisely discusses limitations and
considerations of the results, and their consequences. & Discusses
limitations and considerations of the results and their consequences,
but misses some big considerations. & Discusses limitations and
considerations of the results, but does not discuss the consequences. &
Discusses limitations and considerations of the results, but misses many
considerations and does not discuss consequences. & No discussion
included. \\
Conclusion and References & For the conclusion, main research question
is answered and statistical caveats described to non-technical person.
References are mostly cited consistently within the report, and in the
Reference section. This includes the data source! & For the conclusion,
main research question is answered and statistical caveats described to
non-technical person. References are sometimes cited consistently within
the report, and in the Reference section. This includes the data source!
& For the conclusion, main research question is somewhat answered (but
focus is not on the research question) and statistical caveats described
to non-technical person. References are sometimes cited consistently
within the report, and in the Reference section. This includes the data
source! & For the conclusion, main research question is somewhat
answered (but not the focus at all) and statistical caveats are not
described. References are not cited consistently within the report, and
in the Reference section. This includes the data source! & For the
conclusion, main research question is not answered. Or references are
not included at all. \\
Reflection & Discusses all the labs. Reflection demonstrates an
understanding of each lab's purpose and overarching connection from the
labs to overall project. & Discusses 3 out of 4 labs. & Discusses 2 out
of 4 labs. & Discusses 1 out of 4 labs. & No reflection included \\
\end{longtable}

\subsubsection{More clarifications}

\begin{itemize}
\item
  In formatting, an example of a report with little editing needed is
  one that has zero to some grammar or spelling mistakes, no code chunks
  showing, and no output warnings nor messages showing.
\item
  Professional figures mean

  \begin{itemize}
  \item
    I can read the words and numbers in the html

    \begin{itemize}
    \item
      Variable names are converted from the data frame version to
      readable text
    \item
      For example: \texttt{iam\_001} does not show up on axes, instead
      something like: \texttt{Response\ to\ "Currently,\ I\ am..."}
    \end{itemize}
  \item
    Colors are only used if conveying information
  \item
    Intended message of the figure is easily understood

    \begin{itemize}
    \tightlist
    \item
      If you are trying to show a trend of mean IAT vs.~an ordered
      categorical variable, then the variable is ordered on the x-axis
    \end{itemize}
  \end{itemize}
\item
  For the references

  \begin{itemize}
  \item
    I will not be overly critical about the formatting
  \item
    By consistency, I mean that you if you are citing things like (Last
    Name, Year) it doesn't suddenly change to number citations.
  \item
    If you would like to use Quarto's citation tool, you can! I actually
    pair it with Zotero and it works beautifully! (But I would not
    embark on this if you haven't used Zotero before)
  \end{itemize}
\end{itemize}

\hypertarget{sections}{%
\subsection{Sections}\label{sections}}

\hypertarget{title}{%
\subsubsection{Title}\label{title}}

\begin{itemize}
\tightlist
\item
  \textbf{Purpose:} Create an identifiable name for your research
  project that includes the main research question's variables and gives
  some context to the analysis or results
\end{itemize}

\hypertarget{introduction}{%
\subsubsection{Introduction}\label{introduction}}

\begin{itemize}
\tightlist
\item
  \textbf{Length: 1-2 paragraphs}
\item
  \textbf{Purpose:} Introduce the project motivation, data, and research
  question. It also includes any background information relevant for
  understanding the analysis and relevant previous work.
\item
  This section is non-technical.

  \begin{itemize}
  \tightlist
  \item
    By reading just the introduction, someone without a technical
    background should have an idea of what they study was about, and why
    it is important
  \end{itemize}
\item
  {\textbf{You may start with the introduction written in Lab 1, but you
  should edit it and make sure it flows into your report well!}}
\item
  Should contain some references
\item
  Should include a sentence that states your research question (but NOT
  using a question). For example: ``This study investigates the
  association between food insecurity and age.''
\end{itemize}

\hypertarget{statistical-methods}{%
\subsubsection{Statistical Methods}\label{statistical-methods}}

\begin{itemize}
\tightlist
\item
  \textbf{Length: 3-5 paragraphs}
\item
  \textbf{Purpose:} Describe the analyses that were conducted and
  methods used to select variables and check diagnostics
\item
  \textbf{Important to keep in mind:} methods typically describe your
  approach and process, \emph{not the results of that process}

  \begin{itemize}
  \tightlist
  \item
    For example: I might say ``We investigated the linearity of each
    continuous covariate visually. If continuous variables were not
    linear, then we divided the variable into categories using existing
    guidelines from \textless insert reference here\textgreater{} or
    creating quartiles.''

    \begin{itemize}
    \tightlist
    \item
      In the methods section, I would NOT say: ``We investigated the
      linearity of each continuous covariate visually. We found that age
      was not linearly related to IAT scores. Thus, we categorized age
      into the following groups: \_\_\_, \_\_\_\_, \_\_\_\_, \_\_\_\_,
      and \_\_\_\_.''

      \begin{itemize}
      \tightlist
      \item
        The last two sentences about age would be more appropriate in
        the Results section
      \end{itemize}
    \end{itemize}
  \end{itemize}
\item
  \textbf{Some important methods to discuss} (You may divide these into
  your sections, not necessarily with these names)

  \begin{itemize}
  \tightlist
  \item
    General approach to the dataset

    \begin{itemize}
    \tightlist
    \item
      3-5 sentences
    \item
      Did you need to do any quality control?
    \item
      {\textbf{Missing data: complete case analysis or imputation based
      on Lab 3 choices}}

      \begin{itemize}
      \tightlist
      \item
        1 sentence
      \item
        Can be included in the Exploratory data analysis section
      \end{itemize}
    \end{itemize}
  \item
    Variable transformations

    \begin{itemize}
    \tightlist
    \item
      This includes a description of analyses for Table 1 and what
      statistics were used to summarize the variables

      \begin{itemize}
      \tightlist
      \item
        More on creation of Table 1, not discussing the results of Table
        1
      \end{itemize}
    \item
      Includes (not required)

      \begin{itemize}
      \tightlist
      \item
        Categorizing a continuous variable (even if performed in model
        selection)
      \item
        Did you use family size/household size/number of children as
        categorical or continuous?
      \item
        Did you make any transformations to income?
      \item
        Using scoring for an ordered categorical variable (that is not
        your explanatory variable)
      \end{itemize}
    \item
      1 sentence per \textbf{changed} variable

      \begin{itemize}
      \tightlist
      \item
        Making a categorical variable a factor in our R code is NOT a
        change!
      \end{itemize}
    \end{itemize}
  \item
    Model building: we performed purposeful selection OR LASSO
    regression

    \begin{itemize}
    \tightlist
    \item
      3 sentences max
    \item
      {\textbf{This is where we discuss our process from Lab 4}}
    \item
      For purposeful selection, includes:

      \begin{itemize}
      \tightlist
      \item
        Describe purposeful selection: combining existing literature,
        clinical significance, and analysis
      \item
        How did you build the model? Describe the process in ONE
        SENTENCE
      \item
        Did you consider confounders and effect modifiers?
      \end{itemize}
    \item
      For LASSO:

      \begin{itemize}
      \tightlist
      \item
        Mention that you used LASSO regression with important facts like
        the penalty used, percent testing set, included interactions in
        the selection process
      \end{itemize}
    \end{itemize}
  \item
    Model diagnostics and model fit

    \begin{itemize}
    \tightlist
    \item
      2-5 sentences
    \item
      {\textbf{This is where we discuss our process from Lab 4}}
    \item
      Includes

      \begin{itemize}
      \tightlist
      \item
        Process of investigating model diagnostics and fit
      \item
        If assumptions were not met, what process did you use to fix it?
      \item
        Again, we will NOT discuss the results of the model diagnostics

        \begin{itemize}
        \tightlist
        \item
          For example: ``We investigate the change in Pearson residual
          for every observation. For observations with large changes
          (exceeding change of 4), we check the feasibility of their
          measurements.
        \end{itemize}
      \end{itemize}
    \end{itemize}
  \end{itemize}
\end{itemize}

\hypertarget{results}{%
\subsubsection{Results}\label{results}}

\begin{itemize}
\tightlist
\item
  \textbf{Length: \textasciitilde2-3 paragraphs}
\item
  \textbf{Purpose:} Relay the results from our sample's analysis
  typically focusing on the numbers and interpretations

  \begin{itemize}
  \tightlist
  \item
    The goal is \textbf{not to interpret every single variable} in the
    model but rather to show that you are proficient in using the model
    output to address the research question, using the interpretations
    to support your conclusions.
  \item
    Focus on the variables that help you answer the research question
    and that provide relevant context for the reader.
  \end{itemize}
\item
  Some important results to discuss (also could be sections)

  \begin{itemize}
  \tightlist
  \item
    Sample data set statistics (Table 1)

    \begin{itemize}
    \tightlist
    \item
      3-5 sentences
    \item
      Include a brief description of the sample's characteristics
    \item
      Table 1 should be referenced and appear here!
    \end{itemize}
  \item
    Final model

    \begin{itemize}
    \tightlist
    \item
      1-2 sentences
    \item
      Describe final model (or models if comparing a few)

      \begin{itemize}
      \tightlist
      \item
        What variables were included in your final model?
      \item
        What interactions with your explanatory variable did you
        include?
      \end{itemize}
    \end{itemize}
  \item
    Interpret the model coefficients in the context of the research
    question

    \begin{itemize}
    \tightlist
    \item
      1 paragraph (maybe 2 in special cases)

      \begin{itemize}
      \tightlist
      \item
        When in doubt, ask Nicky if your analysis is a special case
      \end{itemize}
    \item
      Interpreting the explanatory variable's relationship with food
      insecurity is the most important thing to report!!

      \begin{itemize}
      \tightlist
      \item
        When doing this, make sure you account for ALL interactions: If
        your explanatory variable has multiple interactions and you are
        trying to interpret one, then what does that mean about the
        other variables involved in the other interactions? If this is
        confusing, please make an appointment with me!!
      \end{itemize}
    \end{itemize}
  \item
    Results of model diagnostics and model fit if there is anything
    worth noting
  \end{itemize}
\item
  Tables \& figures

  \begin{itemize}
  \tightlist
  \item
    The following are required tables or figures

    \begin{itemize}
    \tightlist
    \item
      Table 1 summarizing participant characteristics both
      \textbf{overall} and \textbf{stratified by your primary
      independent variable}
    \item
      Table or figure with regression results

      \begin{itemize}
      \tightlist
      \item
        Can be a forest plot
      \item
        If you have A LOT of coefficient estimates, the forest plot may
        not work well!
      \end{itemize}
    \end{itemize}
  \item
    \textbf{1-3 figures that you think are helpful in understanding the
    results, for example}

    \begin{itemize}
    \tightlist
    \item
      DAG explaining connection between variables (if you did this)
    \item
      Table or figure to compare model fit statistics (if you did this)
    \item
      Table or figure for unadjusted relationship between outcome and
      explanatory variables
    \end{itemize}
  \end{itemize}
\end{itemize}

\hypertarget{discussion}{%
\subsubsection{Discussion}\label{discussion}}

\begin{itemize}
\tightlist
\item
  \textbf{Length: 2-3 paragraphs}
\item
  \textbf{Purpose:} Discuss the results and give them context outside of
  the sample and its analysis
\item
  Some important things to include

  \begin{itemize}
  \tightlist
  \item
    Include a paragraph on the limitations of the results

    \begin{itemize}
    \tightlist
    \item
      You don't need to hit all the limitations, but think about the big
      ones (generalizability? independence of samples? large sample size
      vs.~clinical significance? the way we handled variables?)
    \end{itemize}
  \item
    After limitations, discuss the positive parts of the results

    \begin{itemize}
    \tightlist
    \item
      What can we do with these results? What impact can it have?
    \end{itemize}
  \item
    Any overarching trends that are worth noting? {[}@Giebel2024{]}
  \end{itemize}
\item
  Should contain some references
\end{itemize}

\hypertarget{reflection}{%
\subsubsection{Reflection}\label{reflection}}

\begin{itemize}
\tightlist
\item
  How did each Lab help you build towards your project report?

  \begin{itemize}
  \tightlist
  \item
    \textbf{Length: 1-2 sentences per lab}
  \end{itemize}
\item
  Did you change anything from your labs?

  \begin{itemize}
  \tightlist
  \item
    \textbf{Length: 1-5 sentences per lab}

    \begin{itemize}
    \tightlist
    \item
      1 sentence to say no changes were made or up to 5 sentences
      describing the changes you made
    \end{itemize}
  \item
    For example, if I gave you feedback about changing a variable to a
    factor, you do not need to discuss that here.
  \item
    However, if you made a serious change to how you built your model
    from Lab 4 (after seeing my feedback) then put that here.
  \end{itemize}
\end{itemize}

\hypertarget{references}{%
\subsubsection{References}\label{references}}

\begin{itemize}
\tightlist
\item
  Include your references here!
\item
  You introduction should have references, especially when discussing
  the social science behind the analysis
\item
  You must reference the IAT data source!!
\end{itemize}

\hypertarget{optional-appendix}{%
\subsubsection{Optional: Appendix}\label{optional-appendix}}

\begin{itemize}
\tightlist
\item
  Additional figures, tables, and plots that you may want to include
\end{itemize}



\end{document}
